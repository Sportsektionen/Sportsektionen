\documentclass[12pt,a4paper]{article}
\usepackage[T1]{fontenc}
\usepackage[swedish]{babel}
\usepackage[utf8]{inputenc}
\usepackage{listings}
\usepackage{graphicx}
\usepackage{color}

\definecolor{green}{rgb}{0,0.6,0}
\definecolor{darkgreen}{rgb}{0.1,0.3,0.1}
\definecolor{mymauve}{rgb}{0.58,0,0.82}

\lstset{ %
  backgroundcolor=\color{white},   % choose the background color; you must add \usepackage{color} or \usepackage{xcolor}
  basicstyle=\footnotesize,        % the size of the fonts that are used for the code
  breakatwhitespace=false,         % sets if automatic breaks should only happen at whitespace
  breaklines=true,                 % sets automatic line breaking
  captionpos=b,                    % sets the caption-position to bottom
  commentstyle=\color{mygreen},    % comment style
  deletekeywords={...},            % if you want to delete keywords from the given language
  escapeinside={\%*}{*)},          % if you want to add LaTeX within your code
  extendedchars=false,              % lets you use non-ASCII characters; for 8-bits encodings only, does not work with UTF-8
  frame=single,                    % adds a frame around the code
  keepspaces=true,                 % keeps spaces in text, useful for keeping indentation of code (possibly needs columns=flexible)
  keywordstyle=\color{blue},       % keyword style
  language=Octave,                 % the language of the code
  morekeywords={*,...},            % if you want to add more keywords to the set
  numbers=left,                    % where to put the line-numbers; possible values are (none, left, right)
  numbersep=5pt,                   % how far the line-numbers are from the code
  numberstyle=\tiny\color{mygray}, % the style that is used for the line-numbers
  rulecolor=\color{black},         % if not set, the frame-color may be changed on line-breaks within not-black text (e.g. comments (green here))
  showspaces=false,                % show spaces everywhere adding particular underscores; it overrides 'showstringspaces'
  showstringspaces=false,          % underline spaces within strings only
  showtabs=false,                  % show tabs within strings adding particular underscores
  stepnumber=1,                    % the step between two line-numbers. If it's 1, each line will be numbered
  stringstyle=\color{mymauve},     % string literal style
  tabsize=2,                       % sets default tabsize to 2 spaces
  title=\lstname                   % show the filename of files included with \lstinputlisting; also try caption instead of title
}


\begin{document}
	% Change to "Årsmöte" or "Annual Meeting" if this is held, or "Extra årsmöte" or "Extra annual meeting".
	\title{\Huge Sektionsmöte Sportsektionen}
	\date{4 februari 2016}
	\maketitle

	\null
	\vfill

	\clearpage

	\begin{enumerate}

		\item Opening of meeting / \emph{Mötets öppnande}
		\item Election of meeting secretary / \emph{Val av mötessekreterare}
		\item Election of meeting chairman / \emph{Val av mötesordförande}
		\item Election of adjustors and tellers / \emph{Val av justerare tillika rösträknare}
		\item Meeting announced in honor of the statutes/ \emph{Fråga om mötets stadgeenliga utlysande}
		\item Adjustment of voting list / \emph{Justering av röstlängd}
		\item Co-opt of present members / \emph{Eventuella adjungeringar}
		\item Determination of agenda / \emph{Fastställande av föredragningslista}
		\item Planning ahead / \emph{Planering framåt}
		\begin{itemize}
			\item Responsibilities and ambitions / \emph{Ansvar och ambitioner}
			\item Transparency and cooperation / \emph{Sammarbete och öppenhet}
			\item Standing out / \emph{``Vilka är Sporten?''}
			\item Building on the legacy / \emph{Vad vill vi ge till dem som tar över?}
			\item Being a representative of the section / \emph{Representera sektionen}
		\end{itemize}
		\item Report from regular events
		\begin{enumerate}
			\item Badminton
			\item Basketball
			\item Floorball
			\item Football
		\end{enumerate}
		\item Follow ups % Add pending items since last meeting
		\begin{enumerate}
			\item Dance - cancelled
		\end{enumerate}
		\item Other topics / \emph{Övriga under mötet väckta frågor}
		\item Push-ups
		\item Ending of meeting / \emph{Mötets avslutande}
	\end{enumerate}
\end{document}
