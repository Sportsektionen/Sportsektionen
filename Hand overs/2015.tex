\documentclass[12pt,a4paper]{article}
\usepackage[T1]{fontenc}
\usepackage[swedish]{babel}
\usepackage[utf8]{inputenc}
\usepackage{lmodern}
\usepackage{listings}
\usepackage{hyperref}
\usepackage{graphicx}
\usepackage{color}

\definecolor{green}{rgb}{0,0.6,0}
\definecolor{darkgreen}{rgb}{0.1,0.3,0.1}
\definecolor{mymauve}{rgb}{0.58,0,0.82}

\lstset{ %
  backgroundcolor=\color{white},   % choose the background color; you must add \usepackage{color} or \usepackage{xcolor}
  basicstyle=\footnotesize,        % the size of the fonts that are used for the code
  breakatwhitespace=false,         % sets if automatic breaks should only happen at whitespace
  breaklines=true,                 % sets automatic line breaking
  captionpos=b,                    % sets the caption-position to bottom
  commentstyle=\color{mygreen},    % comment style
  deletekeywords={...},            % if you want to delete keywords from the given language
  escapeinside={\%*}{*)},          % if you want to add LaTeX within your code
  extendedchars=false,              % lets you use non-ASCII characters; for 8-bits encodings only, does not work with UTF-8
  frame=single,                    % adds a frame around the code
  keepspaces=true,                 % keeps spaces in text, useful for keeping indentation of code (possibly needs columns=flexible)
  keywordstyle=\color{blue},       % keyword style
  language=Octave,                 % the language of the code
  morekeywords={*,...},            % if you want to add more keywords to the set
  numbers=left,                    % where to put the line-numbers; possible values are (none, left, right)
  numbersep=5pt,                   % how far the line-numbers are from the code
  numberstyle=\tiny\color{mygray}, % the style that is used for the line-numbers
  rulecolor=\color{black},         % if not set, the frame-color may be changed on line-breaks within not-black text (e.g. comments (green here))
  showspaces=false,                % show spaces everywhere adding particular underscores; it overrides 'showstringspaces'
  showstringspaces=false,          % underline spaces within strings only
  showtabs=false,                  % show tabs within strings adding particular underscores
  stepnumber=1,                    % the step between two line-numbers. If it's 1, each line will be numbered
  stringstyle=\color{mymauve},     % string literal style
  tabsize=2,                       % sets default tabsize to 2 spaces
  title=\lstname                   % show the filename of files included with \lstinputlisting; also try caption instead of title
}

\hypersetup{
    pdfnewwindow=true,      % links in new PDF window
    colorlinks=true,       % false: boxed links; true: colored links
    linkcolor=red,          % color of internal links (change box color with linkbordercolor)
    citecolor=green,        % color of links to bibliography
    filecolor=magenta,      % color of file links
    urlcolor=blue           % color of external links
}


\begin{document}
	\title{\Huge Transfer Document for Chairman of Sportsektionen - 2015}
	\author{Anton Österberg}
	\date{\today}

	\maketitle

	\null
	\vfill

	\clearpage

	\section{Documents}
	At the start of this year all we had was First Class containing documents from several years back and a link to a Dropbox account to which we had no write access. Since it was decided by DSV that the First Class platform would be shut down during 2015 we had to backup all the files in First Class and migrate to a new platform. We found that the most suitable replacement for the sole purpose of handling files and documents as protocols, reports, statutes, posters and etc. was GitHub which uses version management with the Git protocol. This provides backup since it's hosted by GitHub and provides the transparency we want for our members.

	Some documents of more private or delicate nature, like contracts or data from registration forms which contains personal information is hosted on Google Drive in a shared folder. The ambition is however that as much as possible should be shared and open for the benefit of transparency.

		\subsection{Summary}
		\subsubsection{Documents for Sportsektionen}

		\begin{enumerate}
			\item \emph{First Class} - Deprecated and no longer used. A back up of all documents and discussions from Sportsektionen exists, but have not been processed in any way.
			\item \emph{Dropbox} - Contains many important documents, must of them should however be in our GitHub repository by now.\\
			\url{https://www.dropbox.com/sh/02vawr713yqljjk/AAALsAne3k7C0iWMvi_YVg3La}
			\item \emph{GitHub} - Our primary platform for storage and sharing of most documents, excluding those which contains sensitive data.\\
			\url{https://github.com/mantono/Sportsektionen}
			\item \emph{Google Drive} - Makes up a shared folder that contains important documents that should not be shared with the public.
		\end{enumerate}

		\subsubsection{Documents for DISK}
		\begin{enumerate}
			\item \emph{DISK Website} - At the moment there are only protocols from DISK board meetings and DISK annual meetings from the last year. It is possible there will be more in the future.\\
			\url{http://disk.su.se/dokument}
			\item \emph{GitHub} - This is where the most recent version of the DISK statutes can be found.\\
			\url{https://github.com/StudentkarenDISK/stadga}
		\end{enumerate}

	\section{Platforms}
		We have several different platforms for communication, planning and file management.
		\subsection{GitHub}
			\href{http://github.com}{GitHub} has become out new platform for file storage and management and offers \href{https://help.github.com/articles/what-is-my-disk-quota/}{sufficient storage capacity}. Version management, redundancy, transparency and an interface that offers a good overview has been the main reasons for using GitHub and it is also an established platform in our school of trade since it's using the \href{https://git-scm.com/videos}{Git protocol}.
		\subsection{Slack}
			\href{http://diskdsv.slack.com}{Slack} became necessary after we had to fill the void of First Class as a means of communication. It is a chat platform suitable for organizations like DISK and it prominent features is direct messaging and group chats and as well file sharing to some extent. It consists of channels (public for everyone to join in DISK) and groups (private) as well as private one-on-one conversation. Slack has been used by the DISK board during 2015 and will most likely remain a platform of use for coming boards as well.

			It offers great integration through third-party plugins with services as GitHub, Google Drive and Trello. As of now, every time a task is added to Trello or a new commit is pushed to our GitHub repository we will get a notice to our private group \emph{sportsektionen}.
		\subsection{Office 365}
			\href{http://portal.office.com}{Office 365} became the new platform for e-mail and general discussion (for forum - not chat) and is the main replacement for First Class. The interface is however a bit messy and the use of Office 365 has never been as frequent as First Class ever was. This is used by the DISK board as well as most other sections in DISK, we do not used it a the moment as we have not seen any need for that kind of platform. First Class had the advantage that all students at DSV had a First Class-account, this is not the case with Office 365 were each individual must be signed up manually. The use of First Class in Sportsektionen was quite infrequent even tough everyone had an account. Office 365, which offers about the same functionality but with a less user friendly interface have therefore not seemed like an viable option for us. Adding another platform to our list would only seem like a distraction rather than asset.

			Currently, booking Foo Bar is done through a calendar that is only accessible on Office 365, so for that you would need an Office 365 account. Apart from that, Office 365 is not a very relevant tool for Sportsektionen. It is however good to know about its existence, since it may become mandatory to use for all sections in the future. The person elected as section representative will most likely have to use it in either way.
		\subsection{Trello}
			\href{https://trello.com/b/ez76bD5J/sportsektionen}{Trello} is a simple website for managing simpler to-do lists. It's quite useful when there are a lot of things to take care of and when it's important nothing is forgotten. It has also a good potential for sharing workload between member of the section since it allows collaboration and other members to sign up as responsible for a task. As mentioned before, it has a nice integration with Slack, posting in the group for Sportsektionen every time a new task is created.
		\subsection{Facebook}
			More or less all of our communication with our members of DISK is done through Facebook. Currently we have a Facebook page \href{https://www.facebook.com/sportsektionen/}{DISK Sportsektionen} and several Facebook groups.
			\begin{enumerate}
				\item \href{https://www.facebook.com/groups/246028798885289/}{DISK Sporten} - Our ''main'' group and the most important one.
				\item \href{https://www.facebook.com/groups/327491804118502/}{DISK Innebandy/Floorball}
				\item \href{https://www.facebook.com/groups/487598864634303/}{Basketball in Kista}
				\item \href{https://www.facebook.com/groups/946110802104272/}{Football in Kista}
			\end{enumerate}
			There have also been some plans about creating a bicycle group since there was some interest for that. All tough, I'm not sure more groups are always a good thing since it will make the audience of our events more fragmented.

			The Facebook page is mostly used more for general announcements and to attract new people to Sportsektionen. A page on Facebook has the advantage over a group that it can reach friends of the page followers as were a post in the group can only reach the members of the group. Despite that, a post in the general group ''DISK Sporten'' usually reach more people than a post with the page ''DISK Sportsektionen''. A comparison of the last posts in the group and on the page and their respective reach:
			\begin{enumerate}
				\item Facebook group ''DISK Sporten'' (183 members)
				\begin{enumerate}
					\item Post ''DISK Sporten Dans! (11/12)'' - Seen by 37 (20.2\%) after two days
					\item Post ''DISK Sporten Dans! (4/12)'' (was pinned) - Seen by 56 (30.6\%) after three days
					\item Post for badminton sign-up - Seen by 61 (33.3\%) after four days
					\item Post about the Christmas party for active members - Seen by 67 (36.6\%) after five days
				\end{enumerate}
				\item Facebook page ''DISK Sportsektionen'' (108 likes/followers)
				\begin{enumerate}
					\item Post about DISKs calendar and integration - Seen by 12 (11.1\%) after five days
					\item Post about ''ambassadörsgeneral'' for DISK (contains link) -  Seen by 42 (38.8\%) after six days
					\item Post about ''DISK Sporten Dans! (4/12)'' - Seen by 19 (17.6\%) after seven days
					\item Post about GitHub - Seen by 28 (25.9\%) after twelve days
				\end{enumerate}
			\end{enumerate}

			Their respective average reach for the last four posts are:
			\begin{itemize}
				\item{Facebook group (DISK Sporten)} - 55.25 persons / 30.2\% after 3.5 days
				\item{Facebook page (DISK Sportsektionen)} - 25.25 persons / 23.4\% after 7.5 days
			\end{itemize}

			So at a first glance it might look like the Facebook page is not of much value. But the outcome of the posts and their reach seems to be under the influence of a lot of different variables. The algorithm that affects the reach for pages seems to be way more complex than for a group. Being aware of some of these variables, and using that knowledge right, the reach for a post on the Facebook page can be way more effective than you ever can get in the group. For example, the four posts from the Facebook page that were taken as an example above were all posted with quite short interval, this seems to lower the reach as if there were a maximum quota a pages' post can reach over a certain time.

			Another way to increase the ''popularity'' of a post is tagging other pages - this will make some of the followers of that page an audience to. Sharing images and events seems to boost the reach as well, probably as this will indicate a higher change of having some interesting content for the page followers. Links seems to generate some interest too, but preferably with a substantial amount of text or images as simply posting a lot of links will probably trigger some spam filter. Sharing a status or link from another page seems to get a quite good reach as well, since this uses the authenticity of two different pages; ergo, two pages think this is valuable content to share and this might also be reflected in the page audience.

			To summarize how to get a (very) good reach, follow these guidelines;
			\begin{enumerate}
				\item Do not post too often! Preferably not more than once a week.
				\item Avoid posting content that only contains a small amount of text and nothing else.
				\item Include photos in the post.
				\item Tag other pages.
				\item Share other pages links, posts or events.
				\item Post links, but not as the only content of the post.
				\item Reply to posts made by other persons or pages that are made to the post.
				\item Post content that the audience finds interesting; content that followers are likely to share or comment on.
			\end{enumerate}

			When all these variables are taken into account, you can strike gold! If we look into the most popular post from the last year we can see that this is what enables a good reach.
			\begin{enumerate}
				\item \href{https://www.facebook.com/sportsektionen/posts/797559543632932}{1st of April - 318 persons reached}
				\item \href{https://www.facebook.com/sportsektionen/photos/a.579827052072850.1073741829.577696502285905/878826875506198/?type=3}{28th of September - 267 persons reached}
				\item \href{https://www.facebook.com/sportsektionen/posts/879320665456819}{29th of September - 245 persons reached}
				\item \href{https://www.facebook.com/sportsektionen/posts/816932155029004}{18th of May - 208 persons reached}
			\end{enumerate}
			All of the posts above uses several of the guidelines. Here the potential of the Facebook page is much more obvious. The first post has a reach of 318 persons - that's more than the members of the group and followers of the page combined. At that time the Facebook page had less than 100 followers/likes, still it managed to reach out to three times as many people. More importantly, this kind of contact gives us a broader audience, reaching persons who might not even have heard of Sportsektionen before. This also highlights one of the biggest disadvantages with the Facebook group, what you post in the group never has any potential to reach outside the group.

			With that in mind, it is important to give some thought of what information goes were. Posts that have a better chance to ''go viral'' is probably better suited for the Facebook page, while more formal posts, like announcing time for meetings or similar will most likely get a better reach in the group. With that said, it does not mean that they are mutually exclusive.

		\subsection{Membit}
			Membit is the system DISK uses for member administration. It keeps a record of all members (current and past) and whether they have currently paid for their membership. Access to Membit can be convenient when you want to check for certain that a person is a paying member for DISK, as many events have separate fees for members and non-members or some events (like Åre Skiweek) may only be open to members of DISK.

	\section{Responsibilities \& Tasks}
		\subsection{Booking of Courts}
			One of the (usual) main responsibilities of the chairman is the booking of courts for various activities. \textbf{All bookings must be made together with the chairman of DISK or the cashier of DISK. These two are the only one who are allowed to sign a contract on behalf of DISK.}

			The three different venues that has been used during the previous years are Ärvingehallen, Kista Fire Station and Kista Racket Center.
			\subsubsection{Ärvingehallen}
			Ärvingehallen is booked through an online interface found at \href{http://booking.stockholm.se/}{Stockholms Stad}. When booking, an identification number must be entered, this number can be found at the access card (white) which also gives physical access to Ärvingehallen.

			Since elementary schools has priority on booking, it can be rather difficult to find free time slots on recurring day and time of the week. Ärvingehallen offers a good court with several changing rooms and a high standard, but it's expensive and  it's hard to get a good permanent time, which is why it has become a less viable option.
			\subsubsection{Kista Fire Station}
			Kista Fire Station currently houses most of our activities. It's quite cheap, and it has more available time slots since it's not used by any schools. There are however two major drawbacks with Kista Fire Station. First of all, it only has one dressing and shower room, which is less than ideal when you have participants of both genders. This has been one of the main reasons for looking at other locations.

			The second concern is the capacity, or the lack of it. Currently three activities uses the fire station - floorball, football and basketball -  and all of them have reported that they are from time to time reaching the maximum capacity for it. And while it is cheap compared to Ärvingehallen, it's not so good value for the money considering it's small size and what other venues has to offer.
			\subsubsection{Kista Racket Center}
			Used for badminton and is booked either on location or through telephone (08-750 75 60). Bookings can be made two weeks in advance. It's also possible to play squash and table tennis at Kista Racket Center, but why do that when you can play badminton?
		\subsection{Planning and Organizing Section Meetings}
		Regular meetings can be done at any interval that the board sees fit. Annual meeting must be done once a year, and on top of that, there can be extra annual meetings if required. Annual meetings must be announced no less that 14 days ahead of the meeting, and regular meetings must be announced at least 3 days before the meeting takes place. Extra annual meetings must be announced at least 8 days ahead. The annual meeting may not take place any later that 8 days before DISKs annual meeting. Please see section 6 in the statutes for Sportsektionen for all the details and procedures for meetings.

		Suitable locations for section meetings are group rooms (G10:X, booked in Daisy), seminar rooms (booked via nina@dsv.su.se) and Foo Bar (booked in Office 365).
		\subsection{Participate in Insparken}
		\subsection{Communications \& Information}
			\begin{enumerate}
				\item Website - \href{http://sportsektionen.se}{sportsektionen.se}
				\item DISK Calendar
				\item Skitviktigt
				\item Sportsektionen E-mail
				\item Facebook
			\end{enumerate}
		\subsection{Budget \& Finance}

	\section{Inventories}
		\subsection{Locker in Foo Bar}
		\subsection{Files in Conference Room}
		\subsection{Stowed Items in Basement (\emph{katakomberna})}
			None.


\end{document}
