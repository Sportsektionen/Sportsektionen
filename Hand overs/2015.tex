\documentclass[12pt,a4paper]{article}
\usepackage[T1]{fontenc}
\usepackage[swedish]{babel}
\usepackage[utf8]{inputenc}
\usepackage{lmodern}
\usepackage{listings}
\usepackage{hyperref}
\usepackage{graphicx}
\usepackage{color}

\definecolor{green}{rgb}{0,0.6,0}
\definecolor{darkgreen}{rgb}{0.1,0.3,0.1}
\definecolor{mymauve}{rgb}{0.58,0,0.82}

\lstset{ %
  backgroundcolor=\color{white},   % choose the background color; you must add \usepackage{color} or \usepackage{xcolor}
  basicstyle=\footnotesize,        % the size of the fonts that are used for the code
  breakatwhitespace=false,         % sets if automatic breaks should only happen at whitespace
  breaklines=true,                 % sets automatic line breaking
  captionpos=b,                    % sets the caption-position to bottom
  commentstyle=\color{mygreen},    % comment style
  deletekeywords={...},            % if you want to delete keywords from the given language
  escapeinside={\%*}{*)},          % if you want to add LaTeX within your code
  extendedchars=false,              % lets you use non-ASCII characters; for 8-bits encodings only, does not work with UTF-8
  frame=single,                    % adds a frame around the code
  keepspaces=true,                 % keeps spaces in text, useful for keeping indentation of code (possibly needs columns=flexible)
  keywordstyle=\color{blue},       % keyword style
  language=Octave,                 % the language of the code
  morekeywords={*,...},            % if you want to add more keywords to the set
  numbers=left,                    % where to put the line-numbers; possible values are (none, left, right)
  numbersep=5pt,                   % how far the line-numbers are from the code
  numberstyle=\tiny\color{mygray}, % the style that is used for the line-numbers
  rulecolor=\color{black},         % if not set, the frame-color may be changed on line-breaks within not-black text (e.g. comments (green here))
  showspaces=false,                % show spaces everywhere adding particular underscores; it overrides 'showstringspaces'
  showstringspaces=false,          % underline spaces within strings only
  showtabs=false,                  % show tabs within strings adding particular underscores
  stepnumber=1,                    % the step between two line-numbers. If it's 1, each line will be numbered
  stringstyle=\color{mymauve},     % string literal style
  tabsize=2,                       % sets default tabsize to 2 spaces
  title=\lstname                   % show the filename of files included with \lstinputlisting; also try caption instead of title
}

\hypersetup{
    pdfnewwindow=true,      % links in new PDF window
    colorlinks=true,       % false: boxed links; true: colored links
    linkcolor=red,          % color of internal links (change box color with linkbordercolor)
    citecolor=green,        % color of links to bibliography
    filecolor=magenta,      % color of file links
    urlcolor=blue           % color of external links
}


\begin{document}
	\title{\Huge Transfer Document for Chairman of Sportsektionen - 2015}
	\author{Anton Österberg}
	\date{\today}

	\maketitle

	\null
	\vfill

	\clearpage

	\section{Documents}
	At the start of this year all we had was First Class containing documents from several years back and a link to a Dropbox account to which we had no write access. Since it was decided by DSV that the First Class platform would be shut down during 2015 we had to backup all the files in First Class and migrate to a new platform. We found that the most suitable replacment for the sole puropse of handling files and documents as protocols, reports, statutes, posters and etc. was GitHub which uses version management with the Git protocol. This provides backup since it's hosted by GitHub and provides the transparency we want for our members.

	Some documents of more private or delicate nature, like contracts or data from registration forms which contains personal information is hosted on Google Drive in a shared folder. The ambition is however that as much as possible should be shared and open for the benefit of transparency.

		\subsection{Summary}
		\subsubsection{Documents for Sportsektionen}

		\begin{enumerate}
			\item \emph{First Class} - Deprecated and no longer used. A back up of all documents and discussions from Sportsektionen exists, but have not been processed in any way.
			\item \emph{Dropbox} - Contains many important documents, must of them should however be in our GitHub repository by now.\\
			\url{https://www.dropbox.com/sh/02vawr713yqljjk/AAALsAne3k7C0iWMvi_YVg3La}
			\item \emph{GitHub} - Our primary platform for storare and sharing of most documents, excluding those which contains sensitive data.\\
			\url{https://github.com/mantono/Sportsektionen}
			\item \emph{Google Drive} - Makes up a shared folder that contains important documents that should not be shared with the public.
		\end{enumerate}

		\subsubsection{Documents for DISK}
		\begin{enumerate}
			\item \emph{DISK Website} - At the moment there are only protocols from DISK board meetings and DISK annual meetings from the last year. It is possible there will be more in the future.\\
			\url{http://disk.su.se/dokument}
			\item \emph{GitHub} - This is where the most recent version of the DISK statutes can be found.\\
			\url{https://github.com/StudentkarenDISK/stadga}
		\end{enumerate}

	\section{Platforms}
		We have several different platforms for communication, planning and file management.
		\subsection{GitHub}
			\href{http://github.com}{GitHub} has become out new platform for file storage and management and offers \href{https://help.github.com/articles/what-is-my-disk-quota/}{suffiecient storage capacity}. Version management, redundancy, transparency and an interface that offers a good overiew has been the main reasons for using GitHub and it is also an established platform in our school of trade since it's using the \href{https://git-scm.com/videos}{Git protocol}.
		\subsection{Slack}
			\href{http://diskdsv.slack.com}{Slack} became necessary after we had to fill the void of First Class as a means of communication. It is a chat platform suitable for organizations like DISK and it prominent features is direct messaging and group chats and as well file sharing to some extent. It consists of channels (public for everyone to join in DISK) and groups (private) as well as private one-on-one conversation. Slack has been used by the DISK board during 2015 and will most likely remain a platform of use for coming boards as well.

			It offers great intergration through third-party plugins with services as GitHub, Google Drive and Trello. As of now, every time a task is added to Trello or a new commit is pushed to our GitHub repository we will get a notice to our private group \emph{sportsektionen}.
		\subsection{Office 365}
			\href{http://portal.office.com}{Office 365} became the new platform for e-mail and general discussion (for forum - not chat) and is the main replacment for First Class. The interface is however a bit messy and the use of Office 365 has never been as frequent as First Class ever was. This is used by the DISK board as well as most other sections in DISK, we do not used it a the moment as we have not seen any need for that kind of platform. First Class had the advantage that all students at DSV had a First Class-acccount, this is not the case with Office 365 were each individual must be signed up manually. The use of First Class in Sportsektionen was quite infrequent even tough everyone had an account. Office 365, which offers about the same functionality but with a less user friendly interface have therefore not seemed like an viable option for us. Adding another platform to our list would only seem like a distraction rather than asset.

			It is however good to know about its existence, since it may become mandatory to use for all sections in the future. The person elected as section representative will most likely have to use it in either way.
			% Trello
			% Facebook

	\section{Resposibilites \& Tasks}
		\subsection{Booking of Courts}
			\subsubsection{Ärvingehallen}
			\subsubsection{Kista Fire Station}
			\subsubsection{Kista Racket Center}
		\subsection{Planning and Organizing Section Meetings}
		\subsection{Participate in Insparken}
		\subsection{Communications \& Information}
			\begin{enumerate}
				\item Website - \href{http://sportsektionen.se}{sportsektionen.se}
				\item DISK Calendar
				\item Skitviktigt
				\item Sportsektionen E-mail
				\item Facebook
			\end{enumerate}
		\subsection{Budget \& Finance}

	\section{Inventories}
		\subsection{Locker in Foo Bar}
		\subsection{Files in Conference Room}
		\subsection{Stowed Items in Basement (\emph{katakomberna})}


\end{document}
