\documentclass[12pt,a4paper]{article}
\usepackage[T1]{fontenc}
\usepackage[swedish]{babel}
\usepackage[utf8]{inputenc}
\usepackage{listings}
\usepackage{graphicx}
\usepackage{color}

\definecolor{green}{rgb}{0,0.6,0}
\definecolor{darkgreen}{rgb}{0.1,0.3,0.1}
\definecolor{mymauve}{rgb}{0.58,0,0.82}

\lstset{ %
  backgroundcolor=\color{white},   % choose the background color; you must add \usepackage{color} or \usepackage{xcolor}
  basicstyle=\footnotesize,        % the size of the fonts that are used for the code
  breakatwhitespace=false,         % sets if automatic breaks should only happen at whitespace
  breaklines=true,                 % sets automatic line breaking
  captionpos=b,                    % sets the caption-position to bottom
  commentstyle=\color{mygreen},    % comment style
  deletekeywords={...},            % if you want to delete keywords from the given language
  escapeinside={\%*}{*)},          % if you want to add LaTeX within your code
  extendedchars=false,              % lets you use non-ASCII characters; for 8-bits encodings only, does not work with UTF-8
  frame=single,                    % adds a frame around the code
  keepspaces=true,                 % keeps spaces in text, useful for keeping indentation of code (possibly needs columns=flexible)
  keywordstyle=\color{blue},       % keyword style
  language=Octave,                 % the language of the code
  morekeywords={*,...},            % if you want to add more keywords to the set
  numbers=left,                    % where to put the line-numbers; possible values are (none, left, right)
  numbersep=5pt,                   % how far the line-numbers are from the code
  numberstyle=\tiny\color{mygray}, % the style that is used for the line-numbers
  rulecolor=\color{black},         % if not set, the frame-color may be changed on line-breaks within not-black text (e.g. comments (green here))
  showspaces=false,                % show spaces everywhere adding particular underscores; it overrides 'showstringspaces'
  showstringspaces=false,          % underline spaces within strings only
  showtabs=false,                  % show tabs within strings adding particular underscores
  stepnumber=1,                    % the step between two line-numbers. If it's 1, each line will be numbered
  stringstyle=\color{mymauve},     % string literal style
  tabsize=2,                       % sets default tabsize to 2 spaces
  title=\lstname                   % show the filename of files included with \lstinputlisting; also try caption instead of title
}


\begin{document}
	% Change to "Årsmöte" or "Annual Meeting" if this is held, or "Extra årsmöte" or "Extra annual meeting".
	\title{\Huge Protocol Annual Meeting Sportsektionen}
	\date{\today}
	\maketitle

	\null
	\vfill

	\clearpage

	\begin{enumerate}

		\item Opening of meeting
		
			Meeting opened 11:20.
			
		\item Election of secretary
		
			Anton Österberg is secretary for the meeting.
			
		\item Election of chairman
		
			Marko Saukko is chairman for the meeting.
			
		\item Election of adjusters and tellers
		
			Cristoffer Lagergren and Fredrik Junghem are chosen as adjusters and tellers for the meeting.
			
		\item Meeting announced in honor of the statutes
		
			Meeting announced 14 days prior to the meeting in accordance with the statutes.
			
		\item Adjustment of voting list
		
			All board members present (5)
			\begin{itemize}
				\item Marko Saukko
				\item Cristoffer Lagergren
				\item Anna Vu
				\item Niklas Åkerlund
				\item Max Bertilsson
			\end{itemize}
				
			And two active members (2)
			\begin{itemize}
				\item Anton Österberg
				\item Fredrik Junghem
			\end{itemize}
				
		\item Co-opt of present members
		
			No co-opt needed.
			
		\item Determination of agenda
		
			Two additional items added to the agenda.
			\begin{enumerate}
				\item Responsible for technical manager
				\item Clarification of responsibilities
			\end{enumerate}
			
		\item Election of President and Vice President of Sportsektionen for the next fiscal year
		
			Marko Saukko is elected as president.
			Anna Vu is elected as vice president.
			
		\item Election of board members for the next fiscal year
		
			Niklas Åkerlund, Cristoffer Lagergren, Fredrik Junghem, Max Bertilsson and Anton Österberg are elected as board members.
			
		\item Election of delegate to the DISK board for Sportsektionen
		
			Max Bertilsson is elected as delegate to the DISK board.
			
		\item Election of election committee 
		
			Marko Saukko is elected as election committee.
			
		\item Processing of bills
			\begin{itemize}
				\item Budget for 2017
				
					Minor changes from last years budget. A discussion about adding an additional fee on the price for badminton in order to compensate for the material costs. This would increase the annual earnings from 0 to 640 SEK. The proposed budget will yield a deficit of 210 SEK, compared to last years budget which gave a predicted surplus of 20 SEK. A new post in the budget is Skiweek which has not previously been declared.
					
					Basketball and floorball have seen a decreasing number of participants during the year, and if the trend is continuing, measures will have to be taken in order to not create a too large deficit in the budget.

					The budget is accepted.
			\end{itemize}
		\item Processing of motions
		
			No motions.
			
		\item Other
			\begin{enumerate}
				\item Responsible for technical manager
				
					Anton is elected as technical manager. The role is to oversee and handle maintenance of Github, Slack, sportektionen.se and other technical tools used by the section.
					
				\item Clarification of responsibilities
				
					Anton proposed that we declare clearly in written text in documents what areas of responsibilities we need designated persons for, and that it is clearly described what those responsibilities are. This is mainly to make hand overs easier but it is also meant as a way for us to better understand what roles we need within the section outside managing the sport activities.
					
					Examples of such roles are cashier, secretary, election committee, technical manager and information/PR manager.
					
				\item Bowling

					Short discussion about arranging bowling last Friday pub for the year.
					
			\end{enumerate}
		\item Push-ups done.
		\item Meeting finished 13.50.
	\end{enumerate}
\end{document}
